\documentclass[14pt]{article}
\usepackage{geometry}                % See geometry.pdf to learn the layout options. There are lots.
\geometry{a4paper}                   % ... or a4paper or a5paper or ... 

\usepackage[parfill]{parskip}    % Activate to begin paragraphs with an empty line rather than an indent
\usepackage{fullpage}
\usepackage{graphicx}
\usepackage{grffile}
\usepackage{listings}
\usepackage{hyperref}
\usepackage{tabularx} %tabular with stretch columns
\usepackage{enumerate}
\usepackage{pdfpages}
\usepackage{pdflscape}
\usepackage[utf8]{inputenc}

\usepackage[acronym]{glossaries} % make a separate list of acronyms
\makeglossaries

%testing
% Index
\usepackage{makeidx}
\makeindex

\begin{document}

\begin{titlepage}

\begin{minipage}{4cm}
\begin{tabular}{l}
\includegraphics[width=0.5\textwidth]{Images/uppsala}
\end{tabular}
\end{minipage}
\hfill
\begin{minipage}{4cm}
\begin{tabular}{r}
\includegraphics[width=0.5\textwidth]{Images/logo}
\end{tabular}
\end{minipage}


\begin{center}
% Upper part of the page

\textmd{HealthShare Final Report}
\vfill
% Title
 \huge{\textbf{A Study on Methods to Increase Interoperabilty and Unify Electronic Healthcare Records } }\\[2.0cm]
\begin{center}
for\\
\large\textbf{Uppsala County Council}\\[1.0cm]
by\\
\large{\textbf{Uppsala Universitet}} and \large{\textbf{Rose-Hulman Institute of Technology}}\\[1.0cm]

\end{center}

\begin{center}
\vfill
December 2011\\
\end{center}

\vfill
\begin{center}
This proposal is submitted in partial fulfillment of the requirement of\\
Master of Science in Computer Science\\
in the\\
Department of Computer Science and Engineering\\
Faculty of Engineering\\
University of Moratuwa\\
\end{center}
\vfill

\end{center}

\end{titlepage}

\tableofcontents
\newpage

\begin{abstract}


\newglossaryentry{Electronic Healthcare Record}{name={Electronic Healthcare Record}, description={a systematic collection of electronic health information about individual patients or populations recorded in a digital format that is capable of being shared across different health care settings, by being embedded in network-connected enterprise-wide information systems}}

\newacronym[\glsshortpluralkey=EHRs ,\glslongpluralkey= Electronic Healthcare Records]{EHR}{EHR}{\gls{Electronic Healthcare Record}}

\glspl{EHR} are rapidly expanding in their use and their benefits are well known. Significant reduction in the cost of care, improved quality of care, and improved record keeping are just a few of the benefits of using an \gls{EHR}. It is for these reasons that many medical institutions are rapidly adopting \glspl{EHR}. 

An explosion of innovation has stemmed from this rapid adoption and has produced many different solutions from many different providers. Unfortunately, these solutions tend to store and transmit the information that they collect in formats that are not compatible with each other. In order to maximize the benefit received from these \glspl{EHR} they must be able to share information between different systems and locations.

This paper reports on the various aspects of the creating a framework for \gls{EHR} interoperability.
\end{abstract}

\newpage

\printglossaries

\section*{Temporary section: How to do Glossary Entries}

Look at the {\LaTeX} for this section to see how add glossary entries and how to use them in the text.

\newglossaryentry{interoperability}{name={interoperability}, description={The ability of diverse systems and organizations to work together. Please see section~\ref{sec:interopDefinition} for the meaning within healthcare}}


An \gls{interoperability} entry and \gls{EHR}. Second use: \gls{EHR}.

%Plurals: \glspl{interoperability}. 
Reset acronym\glsreset{EHR}. \\
Plural: First use: \glspl{EHR}. Second use: \glspl{EHR}.

\newpage

\section{Introduction}
Define the problem and in what context (maybe contextualize, write a story, describe a scenario? a concrete relevant scenario that occurs with persons involved)


\subsection{Background}

The setting of problems in \gls{EHR}/EMR \gls{interoperability} situations. Describe what the problems related to \gls{interoperability} are within healthcare. The situation in Uppsala County, why we are doing this. What is being done on \gls{interoperability} in other areas. Describe the problems related to interoperating \glspl{EHR}.
describe the ideal interoperating system

\subsection{Purpose and Scope}
To investigate the \gls{interoperability} of
\gls{EHR}/EMR systems taking in regard what
the people involved want, the limitations,
standards and organizational issues.

Explain what we will look into and why, limit the scope to what we will be discussing in the paper.

\subsection{Reading Instructions}
instructions for different audiences, so people who already know, or don't care about certain sections, don't have to sift through the paper to find information relevant to them. For the rest of the report.

\begin{description}
\item[Administrators] Section * and *
\item[Doctors] Sections * and *
\item[Politicians] Sections * and *
\item[Public] Section * and *
\item[Technicians] Sections * and *
\end{description}

\newpage

\section{Method}
\section{Reserach}
\subsection{Project Organization}
 (How, and our own organization)

\subsection{Interviews}
\subsubsection{Phone interviews}
\subsubsection{Regular interviews}

Based on the relevant literature, interview templates were produced before contact was initiated. The templates were then adapted to fit with the specific individual ranges of knowledge and sent to them beforehand to allow for them to prepare. Contact was initiated through both phone calls and email conversations, to book the time of meeting and sending the interview questions.
\\\\During the actual interview, typically we'd be one or two people interviewing a single person, but this wasn't
\subsubsection{Hybrids between the two}

(preparations, structure, outlines/protocols, focus of the interviews.
How did we prepare the interviews, how were/weren't they structured, transcribed?)

\subsection{Information search}
Reading, methods for searching for information

\newpage

\section{What the people involved want}
(maybe move to introduction if too small)
\begin{itemize}
\item physicians
\item nurses
\item patients
\item administrative personal
in order to make the adoption of a new system as smooth as possible it is important that we make sure that the final system is able to easily integrate with the already existing systems for EHR handling. This can be done in a few different ways, one of the main ways of doing it is to develop a module that would extend the different EHR systems that already exist and are in use.\cite{EPJ2} For Cambio Cosmic there is already such a module made for integration with NPÖ, however this module is not currently used at the Akademiska hospital in Uppsala\cite{EPJ1}.
\\\\
If we decide to implement an extension module for the sharing of EHRs it is important to make sure that the look and feel are similar to the system that already exists. This is crucial in order make the system popular. 
\\\\
A system for EHR sharing could also be made as an external system, however if this is done Susanne stresses the need for it to still be somewhat integrated with the existing system. For example this could be done by implementing an exporting feature into the already existing system that would export the information of the active patient into the new system. This would simplify the use of the new system since it would eliminate the need of inputting the patient information multiple times.
\\\\
Another very important issue to consider is the need to be able to remove certain information about a particular patient. For example, a patient who have received psychiatric care might not want that information to be left on the the patients EHR after treatment have been completed\cite{EPJ1}.
\item the public
\item politicians
\item technicians
\item legal authorities
\item lab workers
\end{itemize}

\newpage

\section{Interoperability}

\subsection{Introduction} % Introduction to what this chapter should be about. (relevant, not too long, not too short) ...Should this really be a subsection or just text?

\subsection{Definition within healthcare} %Define what \gls{interoperability} is. Talk about \gls{interoperability} in general, and also in this specific case. Why \gls{interoperability} is important and how it works.
\label{sec:interopDefinition}

\subsection{Current situation} %What the current situation is regarding \gls{interoperability}. Why it is a big problem.
Today in Sweden the electronic interoperability among hospitals is limited. This is also very true for the sharing of EHRs. The sharing of EHRs today are done in mainly two ways.
\\\\
One of them works in such a way that if an EHR needs to be sent from one hospital to another one, the first hospital makes a print out of the EHR and then anonymize this one before the document is sent via fax to the other hospital. When the second hospital receives the document contact is made between the hospitals and the identifying information is communicated to the second hospital.\cite{EPJ2} The reason for anomyzation is to make sure that even if a fax is accidently sent to the wrong address it will not be possible to distinguish which patient the information is relevant to.
\\\\
The second way that a transfer of an EHR between hospitals is conducted today is that the EHR is sent with the patient during transport, for example if a patient needs to be moved from Uppsala to Stockholm the patient and the corresponding EHR can be sent on the same ambulance to Stockholm\cite{EPJ2}. Problems with this is that it can become stressful during a patient transportation since it can become a time issue, since the writing of the summary might have to be done while the ambulance is ready to go\cite{EPJ2}.
\\\\
A lot of work is done to increase the interoperability, for example the National strategy for eHealth Sweden, which was launched in 2006 aims to increase the interoperability by bringing laws and regulations in Sweden in line with the extended use of information and communication technologies (ICT). It also aims to facilitate interoperable ICT systems and make access to information across organisational boundaries\cite{NationalStrategy}. 
\\\\
Currently the main initiative for the electronic sharing of patient records that are in used today is something called NPÖ, "Nationell Patient Översikt" ("National Patient Summary"). NPÖ is a project that was started during 2004 by what was then called Carelink\cite{ViktorJernelov}. The goal of the NPÖ-system is to provide the different caregivers, such as hospitals with mirrored information from the different EHR systems \cite{ViktorJernelov}. It works in such a way that when a caregiver wants to connect with the NPÖ system the caregiver can get information about a patient via a web service. 
\\\\
One of the main technical problems with the NPÖ-system is the removal of information from the system. As it is today if you remove some part of information from the local EHR and you also want the same information removed from the NPÖ-system you have to do this manually.
\\\\
Many patients today expect a high level of electronic interoperability when it comes to their patient records\cite{EPJ2}, something that is not really there today. The reason for that is assumed might be a lack of familiarity of the current laws in place as well as thinking that patient data is shared between the different counties. This means that there is a risk of misunderstandings if the patient assumes that their patient records are accessible through every hospital in Sweden even though they might just be able to get a summary through the NPÖ-system.

\section{Technicians}

\end{document}



\subsection{Prerequisites for interoperability on a national level}

To achieve European wide solutions for cross-border transfer of patient information certain requirements need to be fulfilled. There has to be an agreement on the definitions of data sets for both patient summaries and e-prescription, a legal framework for data transfer, a technical framework to connect the systems at each level and a working semantic \gls{interoperability}. \cite{epSOS1}

\subsection{Standardized care plans}

%Subsections on different \gls{interoperability} solutions.
\subsection{Solution 1}
\subsection{Solution 2}
\subsection{Solution 3}

\newpage

\section{Technical Standards}

\subsection{What is a Technical Standard for Computer Software?}
A technical standard is a recognized and established requirement about software systems that establishes the ``way things should be done." They are responsible for everything from the uniformity of web browsers (although not all browsers strictly conform to the standard) to the ability for nearly any laptop to connect wirelessly to any access point. Such standards specify aspects of a program such as a common format for a file or data transfer that allow different developers to develop separate software programs while allowing for \gls{interoperability} between the different systems. Software standards are the foundation for the \gls{interoperability} of different software systems.

\subsection{The Creation of a Standard}
Typically for different parties (software development companies) to agree on a specific software standard they create a software standards organization that consist of members and representatives of the various software companies who contribute ideas and opinions about making a single, unified standard addressing the data \gls{interoperability} problem that needs to be addressed.

\subsubsection{ Examples of Software Standards Organizations}

\begin{itemize}
\item World Wide Web Consortium (W3C) - Responsible for web standards such as HTML, HTTP, and XML 
\item Institute of Electrical and Electronics Engineers Standards Association (IEEE Standards Association) - Responsible for a wide range of standards for engineering such as the 802.11 standard
\item Internet Engineering Task Force (IETF) - Responsible for providing standards for new Internet related technology 
\end{itemize}

\subsection{ Adherence to Standards}
Adherence to a particular software standard can be either mandatory required or voluntarily followed. For example, in the United States banking software must conform to security standards and regulations set forth by the Federal Deposit Insurance Corporation (FDIC). Any system that does not conform to these standards faces legal action by the federal government and as such is not allowed for use by U.S. banks. On the other end of the spectrum, the standards set forth by the W3C as to how a web browser should interpret and render various web pages is a completely voluntary standard and while most popular, modern browsers do conform in the most part to these standards there are many cases where some (if not all) do not completely achieve compliance. However, they face no legal retaliation for not complying with the W3C's standards as the standards are voluntarily adhered to and not legally mandated.

\subsection{Software Standards and EHR Software}
With the growing ubiquitousness of the Internet, computers, and modern technology as well as the geographic diversity of medical talent and information it is doubtless that someday a standard specifying the transmission, storage, and format of electronic healthcare records will be developed; the benefit of such a standard is too great to be ignored for long. Such a standard will require and large organization to oversee and develop as patent data is widely varied and medical breakthroughs are discovered daily requiring such a standard to be constantly updated and modified to stay relevant. 

\subsection{Existing Standards in EHR Software}

\subsection{Ideal EHR Standard}

\subsection{The Alternative to a Future Standard For EHR Software}
The only other option for sharing healthcare records would be to have a unified software system doctors all over the world would use to view and modify patient records. However, such a solution is unrealistic as for one there are already too many \gls{EHR} software companies to reasonably plan for one, new company to globally take over the entire electronic healthcare record industry. Furthermore, with anti-trust laws as well as fair competition clauses in many countries (such as the United States and Sweden) a single organization would not be legally allowed to be the sole producer of \gls{EHR} software systems.



\newpage

\section{Adoption of future systems}

%Current limitations and obstacles

\subsection{Technical}
Security? Limitations for what that can technically be done?


\subsection{Legal}
What does the laws allow? 

The new Swedish legislation provides new opportunities, but not obligations. \cite{RiR19}

\subsection{Organizational}

== current organizational issues that limit the adoption of interoperating systems. ==

%Here are related items from the Empirica interview
\begin{itemize}
\item When it comes to \gls{interoperability}, technical issues and standard specifications are subordinated to legal and organizational issues. The organizational issues include common understanding from people involved in the process and getting the commitment from doctors and nurses. 
\item Clear objectives and strong project leaders with the backing of the top level management and politics are very important factors of success for \gls{EHR} projects. Continuous development and the ability to revise things are also crucial.
\item It is important to reach all the layers of the health care organization to achieve change, to engage the different groups of doctors, nurses, secretaries, technicians and not only have the software developers talk to the management section of the organization.
\item Electronic systems are changing the way doctors work and what responsibilities they have.
\end{itemize}

\newpage

\section{Results/ Discussion}

\subsection{Prerequisites for increased interoperability}

\subsection{Types of solutions and implications for the end user}

\newpage

\section{Conclusions}


\newpage

\begin{appendix}


\end{appendix}

\newpage

\bibliographystyle{unsrt} 
\bibliography{ourBib} 

\end{document} 
