\documentclass[12pt]{article}
\usepackage{geometry}                % See geometry.pdf to learn the layout options. There are lots.
\geometry{letterpaper}                   % ... or a4paper or a5paper or ... 

\usepackage[parfill]{parskip}    % Activate to begin paragraphs with an empty line rather than an indent
\usepackage{fullpage}
\usepackage{graphicx}
\usepackage{grffile}
\usepackage{listings}
\usepackage{hyperref}
\usepackage{tabularx} %tabular with stretch columns
\usepackage{enumerate}
\usepackage{pdfpages}
\usepackage{pdflscape}
\usepackage[utf8]{inputenc}


% Index
\usepackage{makeidx}
\makeindex

\begin{document}


\nocite{Sh:1}

\begin{titlepage}

\begin{minipage}{4cm}
\begin{tabular}{l}
\includegraphics[width=0.5\textwidth]{Images/uppsala}
\end{tabular}
\end{minipage}
\hfill
\begin{minipage}{4cm}
\begin{tabular}{r}
\includegraphics[width=0.5\textwidth]{Images/logo}
\end{tabular}
\end{minipage}


\begin{center}
% Upper part of the page

\textmd{HealthShare Final Report}
\vfill
% Title
 \huge{\textbf{A Study on Methods to Increase Interoperabilty and Unify Electronic Healthcare Records } }\\[2.0cm]
\begin{center}
for\\
\large\textbf{Uppsala County Council}\\[1.0cm]
by\\
\large{\textbf{Uppsala Universitet}} and \large{\textbf{Rose-Hulman Institute of Technology}}\\[1.0cm]

\end{center}

\begin{center}
\vfill
December 2011\\
\end{center}

\vfill
\begin{center}
This proposal is submitted in partial fulfillment of the requirement of\\
Master of Science in Computer Science\\
in the\\
Department of Computer Science and Engineering\\
Faculty of Engineering\\
University of Moratuwa\\
\end{center}
\vfill

\end{center}

\end{titlepage}

\tableofcontents
\newpage

\begin{abstract}
Our abstract goes here... 
\end{abstract}

\newpage


\section{Introduction}
Define the problem and in what context (maybe contextualize, write a story, describe a scenario? a concrete relevant scenario that occurs with persons involved)


\subsection{Background}

Romel hah The setting of problems in EHR/EMR interoperability situations. Describe what the problems related to interoperability are within healthcare. The situation in Uppsala County, why we are doing this. What is being done on interoperability in other areas. Describe the problems related to interoperating EHR.
describe the ideal interoperating system

\subsection{Purpose and Scope}
To investigate the interoperability of
EHR/EMR systems taking in regard what
the people involved want, the limitations,
standards and organizational issues.

Explain what we will look into and why, limit the scope to what we will be discussing in the paper.

\subsection{Reading Instructions}
instructions for different audiences, so people who already know, or don't care about certain sections, don't have to sift through the paper to find information relevant to them. For the rest of the report.


\subsection{Project Organization}
 (How, and our own organization)

\subsection{Interviews}
(preparations, structure, outlines/protocols, focus of the interviews)
\begin{itemize}
\item phone interviews
\item regular interviews
\item hybrids between the two
\end{itemize}

How did we prepare the interviews, how they where/wasn't structured, transcibed(?).

\subsection{Reading}
methods for searching for information

\subsection{Other?}

\section{What the people involved want}
(maybe move to introduction if too small)
\begin{itemize}
\item physicians
\item nurses
\item patients
\item administrative personal
\item the public
\item politicians
\item technicians
\item legal authorities
\item lab workers
\end{itemize}


\section{Interoperabilty}

\begin{itemize}
\item Introduction to what this chapter should be about. (relevant, not too long, not too short)
\item Define what interoperability is. Talk about interoperability in general, and also in this specific case. Why interoperability is important and how it works.
\item What the current situation is regarding interoperability. Why it is a big problem.
\item Prerequisites for interoperability on a national level

To achieve European wide solutions for cross-border transfer of patient information certain requirements need to be fulfilled. There has to be an agreement on the definitions of data sets for both patient summaries and e-prescription, a legal framework for data transfer, a technical framework to connect the systems at each level and a working semantic interoperability. \cite{epSOS1}

\item Subsections on different interoperability solutions.
\item Standardized care plans
\end{itemize}


\section{Current limitations and obstacles for adoption of future systems}

\subsection{Technical}
Security? Limitations for what that can technically be done?


\newpage

\section{Technical Standards}


\subsection{What is a Technical Standard for Computer Software?}
A technical standard is a recognized and established requirement about software systems that establishes the ``way things should be done." They are responsible for everything from the uniformity of web browsers (although not all browsers strictly conform to the standard) to the ability for nearly any laptop to connect wirelessly to any access point. Such standards specify aspects of a program such as a commont format for a file or data transfer that allow different developers to develop separate software programs while allowing for interoperability between the different systems. Software standards are the foundation for the interoperability of different software systems.

\subsection{The Creation of a Standard}
Typically for different parties (software development companies) to agree on a specific software standard they create a software standards organization that consist of members and representatives of the various software companies who contribute ideas and opinions about making a single, unifed standard addressing the data interoperability problem that needs to be addressed.

\subsubsection{ Examples of Software Standards Organizations}

\begin{itemize}
\item World Wide Web Consortium (W3C) - Responsible for web standards such as HTML, HTTP, and XML 
\item Institute of Electrical and Electronics Engineers Standards Association (IEEE Standards Association) - Responsible for a wide range of standards for engineering such as the 802.11 standard
\item Internet Engineering Task Force (IETF) - Responsible for providing standards for new internet related technology 
\end{itemize}

\subsection{ Adherance to Standards}
Adherance to a particular software standard can be either manditorally required or volunteerly followed. For example, in the United States banking software must conform to security standards and regulations set forth by the Federal Deposit Insurance Corporation (FDIC). Any system that does not conform to these standards faces legal action by the federal government and as such is not allowed for use by U.S. banks. On the other end of the spectrum, the standards set forth by the W3C as to how a web browser should interperet and render various webpages is a completely voluntary standard and while most popular, modern browsers do conform in the most part to these standards there are many cases where some (if not all) do not completely achieve complience. However, they face no legal retaliation for not complying with the W3C's standards as the standards are volunteerly adhered to and not legally mandated.


\subsection{Software Standards and EHR Software}
With the growing ubiquitousness of the internet, computers, and modern technology as well as the geographic diversity of medical tallent and information it is doubtless that someday a standard specifying the transmission, storage, and format of electronic healthcare records will be developed; the benefit of such a standard is too great to be ignored for long. Such a standard will require and large organization to oversee and develop as patent data is widely varried and medical breakthroughs are discovered daily requiring such a standard to be constantly updated and modified to stay relevent. 

\subsection{The Alternative to a Future Standard For EHR Software}
The only other option for sharing healthcare records would be to have a unified software system doctors all over the world would use to view and modify patient records. However, such a solution is unrealistic as for one there are already too many EHR software companies to resonably plan for one, new company to globally take over the entier electronic healthcare record industry. Furthermore, with anti-trust laws as well as fair competition clauses in many countries (such as the United States) a single organization would not be legally allowed to be the sole producer of EHR software systems.



\subsection{Legal}
What does the laws allow? 

The new Swedish legislation provides new opportunities, but not obligations. \cite{RiR19}

\subsection{Organizational}
current organizational issues that limit the adoption of interoperating systems


\section{Results/ Discussion}

\subsection{ Prerequisites for increased interoperability}

\subsection{Types of solutions and implications for the end user}


\section{Conclusions}


\newpage

\begin{appendix}


\end{appendix}

\newpage

\bibliographystyle{unsrt} 
\bibliography{ourBib} 

\end{document} 
