\documentclass[12pt]{article}
\usepackage{geometry}                % See geometry.pdf to learn the layout options. There are lots.
\geometry{letterpaper}                   % ... or a4paper or a5paper or ... 

\usepackage[parfill]{parskip}    % Activate to begin paragraphs with an empty line rather than an indent
\usepackage{fullpage}
\usepackage{graphicx}
\usepackage{grffile}
\usepackage{listings}
\usepackage{hyperref}
\usepackage{tabularx} %tabular with stretch columns
\usepackage{enumerate}
\usepackage{pdfpages}
\usepackage{pdflscape}
\usepackage[utf8]{inputenc}


% Index
\usepackage{makeidx}
\makeindex

\begin{document}

\nocite{Sh:1}

\begin{titlepage}

\begin{minipage}{4cm}
\begin{tabular}{l}
\includegraphics[width=0.5\textwidth]{Images/uppsala}
\end{tabular}
\end{minipage}
\hfill
\begin{minipage}{4cm}
\begin{tabular}{r}
\includegraphics[width=0.5\textwidth]{Images/logo}
\end{tabular}
\end{minipage}


\begin{center}
% Upper part of the page

\textmd{HealthShare Final Report}
\vfill
% Title
 \huge{\textbf{A Study on Methods to Increase Interoperabilty and Unify Electronic Healthcare Records } }\\[2.0cm]
\begin{center}
for\\
\large\textbf{Uppsala County Council}\\[1.0cm]
by\\
\large{\textbf{Uppsala Universitet}} and \large{\textbf{Rose-Hulman Institute of Technology}}\\[1.0cm]

\end{center}

\begin{center}
\vfill
December 2011\\
\end{center}

\vfill
\begin{center}
This proposal is submitted in partial fulfillment of the requirement of\\
Master of Science in Computer Science\\
in the\\
Department of Computer Science and Engineering\\
Faculty of Engineering\\
University of Moratuwa\\
\end{center}
\vfill

\end{center}

\end{titlepage}

\tableofcontents
\newpage

\begin{abstract}
Our abstract goes here...
\end{abstract}

\newpage


\section{Introduction}
Define the problem and in what context (maybe contextualize, write a story, describe a scenario? a concrete relevant scenario that occurs with persons involved)

\subsection{Background}
The setting of problems in EHR/EMR interoperability situations. Describe what the problems related to interoperability are within healthcare. The situation in Uppsala County, why we are doing this. What is being done on interoperability in other areas. Describe the problems related to interoperating EHR.
describe the ideal interoperating system

\subsection{Purpose and Scope}
To investigate the interoperability of
EHR/EMR systems taking in regard what
the people involved want, the limitations,
standards and organizational issues.

Explain what we will look into and why, limit the scope to what we will be discussing in the paper.

\subsection{Reading Instructions}
instructions for different audiences, so people who already know, or don't care about certain sections, don't have to sift through the paper to find information relevant to them. For the rest of the report.


\section{Method}

\subsection{Project Organization}
 (How, and our own organization)

\subsection{Interviews}
(preparations, structure, outlines/protocols, focus of the interviews)
\begin{itemize}
\item phone interviews
\item regular interviews
\item hybrids between the two
\end{itemize}

How did we prepare the interviews, how they where/wasn't structured, transcibed(?).

\subsection{Reading}
methods for searching for information

\subsection{Other?}

\section{What the people involved want}
(maybe move to introduction if too small)
\begin{itemize}
\item physicians
\item nurses
\item patients
\item administrative personal
\item the public
\item politicians
\item technicians
\item legal authorities
\item lab workers
\end{itemize}


\section{Interoperabilty}

\begin{itemize}
\item Introduction to what this chapter should be about. (relevant, not too long, not too short)
\item Define what interoperability is. Talk about interoperability in general, and also in this specific case. Why interoperability is important and how it works.
\item What the current situation is regarding interoperability. Why it is a big problem.
\item Prerequisites for interoperability on a national level

To achieve European wide solutions for cross-border transfer of patient information certain requirements need to be fulfilled. There has to be an agreement on the definitions of data sets for both patient summaries and e-prescription, a legal framework for data transfer, a technical framework to connect the systems at each level and a working semantic interoperability. \cite{epSOS1}

\item Subsections on different interoperability solutions.
\item Standardized care plans
\end{itemize}


\section{Current limitations and obstacles for adoption of future systems}

\subsection{Technical}
Security? Limitations for what that can technically be done?

\subsection{Legal}
What does the laws allow? 

The new Swedish legislation provides new opportunities, but not obligations. \cite{RiR19}

\subsection{Organizational}
current organizational issues that limit the adoption of interoperating systems


\section{Results/ Discussion}

\subsection{ Prerequisites for increased interoperability}

\subsection{Types of solutions and implications for the end user}


\section{Conclusions}
test Robert

\newpage

\begin{appendix}


\end{appendix}

\newpage

\bibliographystyle{unsrt} 
\bibliography{ourBib} 

\end{document} 
